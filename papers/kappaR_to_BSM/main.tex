\documentclass[11pt]{article}
\usepackage[margin=1in]{geometry}
\usepackage{amsmath, amssymb}
\usepackage{siunitx}
\usepackage{hyperref}
\usepackage{graphicx}

\title{From Curvature--EM Coupling to BSM Parameter Space: A Framework Linking $\kappa_R$ to Dark Photon and Axion Benchmarks}
\author{Dawson Institute Collaboration}
\date{\today}

\begin{document}
\maketitle

\begin{abstract}
We present a conservative, unit-consistent framework to relate laboratory bounds on the curvature--electromagnetism coupling $\kappa_R$ to beyond-Standard-Model (BSM) parameter space. For dark photon kinetic mixing, we identify a robust mapping $\varepsilon_\mathrm{eff} \simeq C_\varepsilon\,(\kappa_R\,\mathcal{R})$, where $\mathcal{R}$ is a characteristic curvature scale and $C_\varepsilon=\mathcal{O}(1)$ encodes UV matching. This mapping is dimensionless and directly quantifies the size of Maxwell-equation modifications in curved backgrounds. For axions, a CP-even $\kappa_R F^2$ operator cannot be equated to the CP-odd $a F\tilde F$ coupling without additional portal assumptions; we therefore provide a clearly labeled, model-dependent parametrization $g_{a\gamma\gamma}^{\rm equiv}\!\simeq C_a\,(\kappa_R\,\mathcal{R})/\Lambda$ that may be used for benchmarks only. We illustrate the framework with representative curvature environments (lab, Earth surface, magnetar), and provide tables/plots for $\varepsilon_\mathrm{eff}$ and $g_{a\gamma\gamma}^{\rm equiv}$ across $C_{\varepsilon,a}$.
\end{abstract}

\section{EFT setup and conventions}
We work with the CP-even, dimension-6 operator
\begin{equation}
\mathcal{L} \supset \kappa_R\, \mathcal{R}\, F_{\mu\nu}F^{\mu\nu},
\end{equation}
where $\kappa_R$ has dimensions of length$^2$ in SI and mass$^{-2}$ in natural units ($\hbar=c=1$). The characteristic curvature scale $\mathcal{R}$ carries length$^{-2}$ (mass$^2$). The product $\kappa_R\,\mathcal{R}$ is therefore dimensionless and controls the fractional correction to the Maxwell kinetic term.

For dark photons, the kinetic-mixing operator is
\begin{equation}
\mathcal{L} \supset -\frac{\varepsilon}{2}\, F_{\mu\nu}F'^{\mu\nu},
\end{equation}
with dimensionless $\varepsilon$. We adopt a conservative identification
\begin{equation}
\varepsilon_\mathrm{eff} \;\simeq\; C_\varepsilon\, (\kappa_R\,\mathcal{R}),
\label{eq:eps-map}
\end{equation}
where $C_\varepsilon\sim\mathcal{O}(1)$ encodes UV matching and spin-1 portal structure.

For axions, the operator is CP-odd and dimension-5,
\begin{equation}
\mathcal{L} \supset \frac{g_{a\gamma\gamma}}{4}\, a\, F_{\mu\nu}\tilde F^{\mu\nu}.
\end{equation}
Any mapping from $\kappa_R$ to $g_{a\gamma\gamma}$ requires additional CP-odd portal(s), e.g. $a\,\mathcal{R}/\Lambda$ or curvature-induced axion backgrounds. We therefore provide a~parametric benchmark
\begin{equation}
 g_{a\gamma\gamma}^{\rm equiv} \;\simeq\; C_a\,\frac{\kappa_R\,\mathcal{R}}{\Lambda},
 \label{eq:axion-map}
\end{equation}
with an explicit UV scale $\Lambda$ (we show examples for $\Lambda=10$~TeV). This should be interpreted as an \\emph{illustrative reach}, not a model-independent constraint.

\section{Curvature environments}
We consider representative curvature scales $\mathcal{R}$ (order-of-magnitude): lab near-flat ($10^{-30}\,\mathrm{m}^{-2}$), Earth surface ($10^{-26}\,\mathrm{m}^{-2}$), and magnetar surfaces ($10^{-6}\,\mathrm{m}^{-2}$). Our code exposes these as presets and allows user-defined inputs.

\section{Results: dark photon kinetic mixing}
Using Eq.~\eqref{eq:eps-map}, we tabulate $\varepsilon_\mathrm{eff}$ for laboratory $\kappa_R$ bounds and the above curvature benchmarks, scanning $C_\varepsilon\in\{1,10^{-2},10^{-4}\}$. For near-flat laboratory curvature, $\varepsilon_\mathrm{eff}$ is tiny; astrophysical environments can enhance the effective mixing by many orders of magnitude. We provide machine-readable tables and optional plots.

\section{Axion benchmarks (model-dependent)}
With Eq.~\eqref{eq:axion-map}, we show $g^{\rm equiv}_{a\gamma\gamma}$ for $C_a\in\{1,10^{-2}\}$ and $\Lambda=10$~TeV. We stress that this does not replace direct axion constraints (which rely on external fields and CP-odd dynamics); instead, it provides a bridge to discuss joint sensitivity in environments where curvature is non-negligible.

\section{Discussion and outlook}
Our framework provides a principled first connection from curvature--EM EFT coefficients to BSM parameter space. It yields dimensionless, conservative comparisons for dark photon mixing, and a transparent, explicitly model-dependent parametrization for axions. Future work will refine the curvature presets with realistic geometries and incorporate propagation effects and birefringence. Code and data are available in the accompanying repository modules.

\paragraph{Artifacts.} We include CSV tables in results/bsm\_bounds and optional figures under figures/bsm\_bounds.

\end{document}
